\documentclass{beamer}

\mode<presentation>
{
  \usetheme{Madrid}      % or try Darmstadt, Madrid, Warsaw, ...
  \usecolortheme{whale} % or try albatross, beaver, crane, ...
  \usefonttheme{serif}  % or try serif, structurebold, ...
  \setbeamertemplate{navigation symbols}{}
  \setbeamertemplate{caption}[numbered]
} 
\usepackage{amsmath}
\usepackage{amssymb}
\usepackage[utf8]{inputenc}
\usepackage[T1]{fontenc}
\usepackage{stmaryrd}
\usepackage{relsize}
\usepackage[bbgreekl]{mathbbol}
\usepackage{amsfonts}
\DeclareSymbolFontAlphabet{\mathbb}{AMSb}
\DeclareSymbolFontAlphabet{\mathbbl}{bbold}
\newcommand{\Prism}{{\mathlarger{\mathbbl{\Delta}}}}

\theoremstyle{plain}
\newtheorem{cor}{Corollaire}[section]
\newtheorem{thm}{Theorem}[section]
\newtheorem{lemme}{Lemma}[section]
\newtheorem{proposition}{Proposition}[section]
\theoremstyle{definition}
\newtheorem{defi}{Definition}[section]
\theoremstyle{remark}
\newtheorem{ex}{Exemple}
\newtheorem{rmk}{Remarque}
\newcommand{\Q}{\mathbb{Q}}
\newcommand{\Z}{\mathbb{Z}}
\newcommand{\an}{A_{K_n}}
\newcommand{\mo}{\mathcal{O}}

\title{Introduction to Iwasawa Theory}
\author{Anlun Li}
\institute{USTC}
\date{February 23, 2022}

\begin{document}
\begin{frame}
  \titlepage
\end{frame}

%Uncomment these lines for an automatically generated outline.
\begin{frame}{Plan}
  \begin{itemize}
    \item Motivations and Backgrounds
    \item Basic Notations and Facts
    \item Iwasawa Main conjecture
    \item An easy application
  \end{itemize}
\end{frame}

\section{Motivations}
\begin{frame}{Notations}
  Let $K/\Q$ be a finite extension. Then we may consider the distance between 
  $O_K$ and PID. We define $Cl(K)$ to be its ideal class group to measure its difference.
  \begin{definition}
    $Cl(K)=\{ \text{Invertible fractional ideal}\}/
     \{\text{ Principal fractional ideal}\}$
     $h_K= \#Cl(K)$
  \end{definition}
  There is a theorem showing that $h_K$ is finite in general. We omit the proof.
\end{frame}


\begin{frame}{Kummer's two propositions}
  In fact, Kummer has developed serveral propositions that makes $h_K$ be powerful.
  \begin{proposition}[Relating to Fermat's Last Theorem]
    If $p \nmid h_{\Q(\mu_p)}$, then $x^p+y^p=z^n$ has no solutions in $\Z^3$.
  \end{proposition}
  \begin{proposition}
    $p \mid h_{\Q(\mu_p)} \ \iff \exists$ positive even integer r, such that 
    $p \mid \zeta(1-r)$
  \end{proposition}
  We will briefly prove the latter proposition at the end of this talk.
\end{frame}

\section{Basic Notations and Facts}
\begin{frame}{Notations}
  Henceforth, we assume p is an odd prime. And \[K:=\Q(\mu_p), K_n:=\Q(\mu_{p^n})
  ,K_{\infty}:=\Q(\mu_{p^{\infty}})=\bigcup_n \Q(\mu_{p^n}).\]
 
  As we mentioned above, it's improtant to discuss the p part of $Cl(K)$.
  In general, we should focus on the p-sylow subgroup of $Cl(K_n)$ .
\vskip 0.2cm
  Let $Cl(K_n)=\an \oplus \an'$, where $\an$ is its p-sylow subgroup.
\end{frame}

\begin{frame}{Maps between $Cl(K_n)$ and $Cl(K_m)$}
  Suppose n>m, then for $x \in \Q(\mu_{p^n})$, we know
  \[ N(x) = \prod_{\sigma \in Gal(K_n/K_m)} \sigma x \in K_m.\]
  Therefore, we have
  \[ N: \ Cl(K_n) \rightarrow Cl(K_m) \]
  \[ [I] \mapsto [N(I)].\]
  Similarily, we can restrict N to $\an$. And these maps define an inverse limit.

  Let $X= \lim_{\leftarrow} \an$. Next we will talk about its structure.
\end{frame}

\begin{frame}{X,$\an$ are $\Z_p[[G]]$ modules}
  Let $G=$Gal$(K_{\infty}/Q)$. Since for any $\sigma \in G$, $\sigma \mu_{p^n}=\mu_{p^n}^{s_n}$, where 
  $s_n \in (\Z/p^n\Z)^{\times}$. And for n>m, $s_m$ is defined by $s_n$.

  Therefore, 
  \[ 
    \begin{split}
      G  \cong &\lim_{\leftarrow} \text{Gal}(K_n/\Q)
         \cong \lim_{\leftarrow} (\Z/p^n\Z)^{\times} \\
         \cong &(\Z/p\Z)^{\times} \times \lim_{\leftarrow} (\Z/p^{n-1}\Z) 
         \cong (\Z/p\Z)^{\times} \times \Z_p.
    \end{split}
    \]
  We write $G=\Delta \times \Gamma$, where $\Gamma \cong \Z_p$, the p adic integer.
\vskip 0.2cm
  Since $\an$ is finite p group, it is $\Z_p$ module. And for $\sigma \in G$, it 
  can act on $Cl(K_n)$ by $\sigma ([I])= [\sigma(I)]$, so does $\an$.
 \vskip 0.2cm
  In Conclusion, $X$ and $\an$ are $\Z_p[[G]]$ modules.
\end{frame}

\begin{frame}{A lemma for Decomposition}
  Here is a lemma to help us decomposite X and $\an$.
  \begin{lemma}
    If R is a commutative ring containing $<\mu_n>$, $\Delta$ is an abelian group,
     with order=n. Let $\widehat{\Delta}= $Hom($\Delta$,R), then  we have the decomposition:
     \[ M= \oplus_{\chi \in \widehat{\Delta}} e_{\chi}M.\]
    \end{lemma}
    \begin{proof}
      Let \[ e_{\chi}= \frac{1}{n} \sum_{\sigma \in \Delta} \chi(\sigma)\sigma^{-1}.\]
      Note that for any $m \in M$, $m=\sum_{\chi} e_{\chi}m$.
    \end{proof}
\end{frame}

\begin{frame}{Decomposition for X and $\an$}
  Since G, $\an$ are $\Z_p[[G]]=\Z_p[\Delta][[\Gamma]]$ mod, they are $\Z_p[\Delta]$ mod.
  Notice that $\widehat{\Delta}=\{\omega^i\}_{0 \le i \le p-2}$,
  Hence, 
  \[ X= \oplus_{0 \le i \le p-2} X^{\omega^i}, \an =\oplus_{0 \le i \le p-2} \an^{\omega^i}. \]
  We can prove that $X^{\omega^i},\an^{\omega^i}$ are indeed $\Lambda=\Z_p[[\Gamma]]$ mod.
\end{frame}

\begin{frame}{$\Lambda \cong \Z_p[[T]]$}
  It is sufficient to prove that 
  \[ \Z_p[\Z/p^n] \cong \Z_p[T]/((1+T)^{p^n}-1)\]
  and 
  \[\lim_{\leftarrow}\Z_p[T]/((1+T)^{p^n}-1) \cong \Z_p[[T]]. \]
  For the former, only need to verify that the map $\bar{1} \to T+1$ is a bijective.\\
  For the latter, we need to use the fact that $\Z_p$ is p-adic complete.

\end{frame}

\begin{frame}{Pseudo-isomorphism and Char(X)}
  An important theorem tells us that X is a finitely generated tortion $\Lambda$ mod.
  So \[ X \thicksim  \Lambda/f_1^{n_1} \oplus \cdots \oplus \Lambda/f_r^{n_r} . \]
  We say $M \thicksim N$, meaning that there exists $\Lambda$ mod 
  $ \phi: M \to N$, such that $\ker(\phi), coker(\phi)$ have finite length as $\Z_p$ mod.
  \begin{definition}
    \[Char(X):=\prod_{i=1}^{r} f_i^{n_i}\]
  \end{definition}
  Note that this definition is independent of the choice of pseudo-isomorphism.

\end{frame}

\begin{frame}{P-adic L function}
  Before we introduce p adic L function, I should mention a proposition proved by 
  Kummer, which states there exists a form of $\zeta$ whcih has a good property in p adic number field.
  \begin{proposition}[Kummer]
    If $n_1,n_2$ are positive integers and $n_1 \equiv n_2 \neq 0 $ \ (mod (p-1)),
    then \[(1-p^{n_1-1})\zeta(1-n_1) \equiv (1-p^{n_2-1})\zeta(1-n_2) \ (mod p).\]
    More generally, if $p-1 \nmid n_1$ and $n_1 \equiv n_2 \ (mod (p-1)p^{n-1})$,
    then \[ (1-p^{n_1-1})\zeta(1-n_1) \equiv (1-p^{n_2-1})\zeta(1-n_2) \ (mod p^n). \]
  \end{proposition}
\end{frame}

\begin{frame}{P-adic L function}
  We should define a function which has good property of continuous, or even
  holomorphic. Thanks to the proposition above, we can define p-adic L function as follows:
  \[ L_p(1-n,\chi) := (1-\chi \omega^{-n}(p)) L(1-n,\chi \omega^{-n})\]
  Using Euler-product we can show that the right hand side is well defined. Since $\Z_{\le 0}$ 
  is dense in $\Z_p$, if we assume $L_p$ function is continuous, then we have defined a function in $\Z_p$.

\end{frame}

\begin{frame}{Properties of P-adic L function}
  Here we list the properties of p adic L function.
  \begin{itemize}
    \item Continuous
    \item P adic holomorphic
    \item Iwasawa power series
  \end{itemize}
  We say a function is p adic holomorphic, means that 
  \[ \forall \alpha \in \Z_p, \exists a_n \in \overline{ \Q_p}, L_p(s,\chi)=
  \sum_{n=0}^{\infty} a_n(s-\alpha )^n, \forall s \in \Z_p\]
  In the next page we will introduce Iwasawa power series.
  
\end{frame}

\begin{frame}{Iwasawa power series}
  Let $ \mo_{\chi}:=\Z_p[\text{Im} \chi]$.
  \begin{theorem}[Iwasawa Theorem]
    \begin{itemize}
      \item $\exists G_{\chi}(T) \in  $ Frac($\mo_{\chi}[[T]]$), such that,
       \[G_{\chi}((1+p)^s-1)=L_p(s,\chi).\]
      \item If the conductor of $\chi \neq 1 $ or $p^n$ \ (n$ \geq 2$) , then 
      $G_{\chi}$ defined above is in  $\mo_{\chi}[[T]]$. 
    \end{itemize}
  \end{theorem}
  For example, $\chi =\omega^i$ satisfies the second condition.
\end{frame}

\section{Iwasawa Main Conjecture}
\begin{frame}{Statement of Main Conjecture}
  Indeed, this main conjecture is a theorem now.
  \begin{theorem}[Iwasawa Main Conjecture]
    Let $X$,$G_{\chi}$ as defined above, then the following two ideals in $\Z_p[[T]] \cong \Lambda$ is equal:
    \[ ( \text{Char}(X^{\omega^i}) ) =( G_{\chi^{1-i}}(T)) .
      \]
  \end{theorem}
  This theorem connects an algebraic structure to an analytic object.
\end{frame}

\section{An easy application}
\begin{frame}{Results followed directly from Main Conjecture}
  If we assume the following proposition is true, then we can prove 
  Kummer's second proposition mentinoed in our motivation section.
  \begin{proposition}[Dudeced from Iwasawa Theory]
    Suppose $1<i<p-1$, i is an odd integer. Then
    \[ 
      \# A_{\Q(\mu_p)}^{\omega^i}=\# \Z_p/ L(0,\omega^{-i}) 
      =\# \Z_p/  L_p(0,\omega^{1-i})  
    = \# \Z_p/ G_{\omega^{1-i}} (0). \]
  \end{proposition}
  \begin{corollary}[Kummer, Herbrand]
    \[ A_{\Q(\mu_p)}^{\omega^i} \neq \varnothing \iff \exists r>0, 1-i \equiv r \ \text{(mod p-1)}, p | \zeta(1-r) .\]
  \end{corollary}
  If we assume the proposition above is true, we can prove the corollary, using basic properties of 
  p-adic L function. 
\end{frame}

\begin{frame}{Proof of the Corollary}
  By the definition of $L_p(s,\chi)$, we can show that 
  \[ \zeta(1-r) \equiv L_p(1-r,\omega^r) \text{(mod p)}.\]
  On the other hand, notice that $Im(\omega^i) \in \Z_p$, and \[G_{\omega^r}(T)=\sum_{n=0}^{\infty}a_nT^n, \text{where}\  a_n \in \Z_p ;  \]
  \[L_p(s,\omega^r)=G_{\omega^r}((1+p)^s-1)=\sum_{n=0}^{\infty}a_n((1+p)^s-1)^n.\]
  Since \[(1+p)^{1-r}-1=\sum_{n=0}^{\infty} p^n \binom{1-r}{n} -1 \equiv 0 \text{(mod p)},\]
  therefore, $\zeta(1-r) \equiv L_p(1-r,\omega^r) \equiv a_0 \equiv L_p(0,\omega^r)$ (mod p). \
  By using the proposition above, we are done.

\end{frame}

\section{References}
\begin{frame}{References}
  \begin{itemize}
    \item \textit{Number Theory II Iwasawa Theory and Automorphic Form}
    \item Lawerence C. Washington, \textit{Introduction to Cyclotomic Fields}
  \end{itemize}
\end{frame}

\begin{frame}
  \begin{center}
    Thank You!
  \end{center}
\end{frame}
\end{document}
 